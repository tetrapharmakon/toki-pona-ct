\documentclass[10pt,a4paper,twoside]{article}

\usepackage[ top=3.5cm
           , bottom=3.5cm
           , left=3.7cm
           , right=4.7cm
           , columnsep=30pt]{geometry}

\usepackage[utf8]{inputenc}
\usepackage[T1]{fontenc}

\usepackage{palatino} 

\usepackage{microtype} 

\usepackage{multicol} 

\usepackage[bf,sf,center]{titlesec} 

\usepackage{fancyhdr} 
\fancyhead[L]{\textsf{\rightmark}} 
\fancyhead[R]{\textsf{\leftmark}} 
\renewcommand{\headrulewidth}{1.4pt} 
\fancyfoot[C]{\textbf{\textsf{\thepage}}} 
\renewcommand{\footrulewidth}{1.4pt} 
\pagestyle{fancy} 

\newcommand{\entry}[4]{\markboth{#1}{#1}\textbf{#1}\ {(#2)}\ \textit{#3}\ $\mid$\ {#4}}  

\usepackage{lipsum}

\begin{document}
\thispagestyle{empty}
\lipsum[3]
\newpage
\section*{Set theory}
\begin{multicols}{2}
% \entry{Word}{hyphenation}{Type}{Definition}
\entry{Aardvark}{ahrd-vahrk}{Noun}{A nocturnal badger-sized burrowing mammal of Africa, with long ears, a tubular snout, and a long extensible tongue, feeding on ants and termites. Also called antbear.}
\end{multicols}
\section*{General topology}
\begin{multicols}{2}
% \entry{Word}{hyphenation}{Type}{Definition}
\entry{Aardvark}{ahrd-vahrk}{Noun}{A nocturnal badger-sized burrowing mammal of Africa, with long ears, a tubular snout, and a long extensible tongue, feeding on ants and termites. Also called antbear.}
\end{multicols}
\section*{Category theory}
\begin{multicols}{2}
% \entry{Word}{hyphenation}{Type}{Definition}
\entry{Aardvark}{ahrd-vahrk}{Noun}{A nocturnal badger-sized burrowing mammal of Africa, with long ears, a tubular snout, and a long extensible tongue, feeding on ants and termites. Also called antbear.}
\end{multicols}


\end{document}